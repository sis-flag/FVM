\documentclass[12pt,a4paper]{article}

\usepackage[UTF8]{ctex}
\usepackage[scale=0.8]{geometry}
\usepackage{latexsym,amsmath,amsfonts,amssymb,mathrsfs,bm}
\usepackage{abstract,appendix,titlesec,titletoc}
\usepackage{diagbox,booktabs,longtable,tabularx}
\usepackage[amsmath,thmmarks,hyperref]{ntheorem}
\usepackage{fancyhdr,indentfirst}
\usepackage[colorlinks=ture]{hyperref}
\usepackage{makeidx,cleveref}
\usepackage[sort&compress, numbers]{natbib}
\usepackage{graphicx,epsfig,subfig}
\usepackage{algorithm,algorithmicx,algpseudocode}
\usepackage{xcolor}

%\setlength{\lineskip}{\baselineskip}
\setlength{\parskip}{0.5\baselineskip}

\newtheorem{definition}{定义}
\newtheorem{example}{例}
\newtheorem{theorem}{定理}
\newtheorem{lemma}{引理}
\newtheorem{corollary}{推论}
\newtheorem{remark}{注}
\newtheorem{proposition}{性质}

\title{有限体积法代码细节}
\author{sis-flag}
\date{\today}

\begin{document}

\maketitle

区域$\Omega$是二维多边形区域。我们要在区域上求解稳态扩散问题
\begin{align*}
- \nabla \cdot (a \, \nabla u) = f \quad x \in \Omega \\
u = u_0 \quad x \in \Gamma_D \\
a \nabla u \cdot n = g \quad x \in \Gamma_N
\end{align*}
其中$u(x)$是未知函数,$a(x)$是$2 \times 2$对称正定的矩阵,表示扩散系数。

\section*{非结构网格}

非结构网格指的是不规则的多边形网格。二维网格中的要素有:节点(Point),边(Edge),单元(Unit)。三维的情况下,网格为多面体,要素有:节点(Point),边(Edge),面(Face),体(Cell)。这里只考虑二维情况。

理论上来说,想要记录二维网格的所有信息,只需要记录所有点的坐标,和每个单元的顶点是哪些(逆时针排序)就可以了。我们在保存网格信息的时候,也只需要把这两个数组写入文件。

在编写数值格式的程序时,我们还需要计算网格更多的信息。数值格式中需要用到的属性有:每个点的坐标,每条边的中点坐标、边长和法向量,每个单元的中心坐标和面积,以及记录三者之间相互连接关系的表。具体每个类中具有的属性如下:

\begin{tabular}{p{0.3\textwidth}p{0.3\textwidth}p{0.3\textwidth}}
\hline
 Point (节点) & Edge (边) & Unit (单元) \\
\hline
xp, yp (坐标) & xe, ye (middle 中点坐标) & xc, yc (center 中心坐标) \\
 & nx, ny (法向单位向量) & \\
 & len (length 边的长度) & area (单元的面积) \\
nE (neighbor edges 与之相连的边的编号) & nP (neighbor points 与之相连的节点的编号) & nP (neighbor points 与之相连的节点的编号) \\
nU (neighbor units 与之相连的单元的编号) & nU (neighbor units 与之相连的单元的编号) & nE (neighbor edges 与之相连的边的编号) \\
isbdp (is boundary point 布尔型 是否边界点) & & isbdu (布尔型 单元是否靠近边界) \\
\hline
\end{tabular}

某条边是否是边界上的边不用存储,直接看它相邻单元的个数即可。

边的法向量都是由$nU[1]$指向$nU[2]$的,如果在边界上,法向量默认指向区域外。同时,边的法向量也相当于由$nP[1]$指向$nP[2]$的方向向量顺时针旋转90度的方向。

计算单元面积用的是这里的公式:\url{https://zhuanlan.zhihu.com/p/110025234}

判断某一点是否在单元内部,用的是这里的公式\url{https://www.cnblogs.com/luxiaoxun/p/3722358.html}。

\section*{九点格式}

在单元$K$上,九点格式表示为
\begin{align*}
\sum_{e \in nE(K)} \mathcal{F}_{K, e} = |K| \, f(K)
\end{align*}
因此九点格式最终要求解的是一个$Units.len$维的线性方程组,右端项就是$|K| \, f(K)$。

在程序中,我们要遍历每条边来装配总体的系数矩阵。

边$e = AB$是一条位于内部的边,它的法向为$n_e$,从单元$K$指向单元$L$。

边$e$上沿着$n_e$方向的流量为在两个单元上计算出的流量平均
\begin{align*}
\mathcal{F}_{e} = \frac12 (F_{K, e} + F_{L, e})
\end{align*}

在$K$单元上计算出的流量为
\begin{align*}
F_{K, e} = |e| \; (\alpha_{K,A} \, (u(K) - u(A)) + \alpha_{K,B} \, (u(K) - u(B)))
\end{align*}
其中的系数$\alpha_{K,A}, \alpha_{K,B}$由这个方程给出
\begin{align*}
a(K)^T n_{e} = \alpha_{K,A} \, \overrightarrow{KA} + \alpha_{K,B} \, \overrightarrow{KB}
\end{align*}
在$L$单元上计算流量方法类似。其中的系数$\alpha_{K,A}, \alpha_{K,B}$由这个方程给出。注意这里$n_e$的方向没变
\begin{align*}
a(L)^T n_{e} = \alpha_{L,A} \, \overrightarrow{LA} + \alpha_{K,B} \, \overrightarrow{LB}
\end{align*}

总的流量表达式为
\begin{align*}
F_{e} = \frac{|e|}{2} \; \left(\alpha_{K,A} \, (u(K) - u(A)) + \alpha_{K,B} \, (u(K) - u(B)) + \alpha_{L,A} \, (u(L) - u(A)) + \alpha_{L,B} \, (u(L) - u(B))\right)
\end{align*}
注意$n_e$上的流量既是单元$K$流出的,也是$L$流入的,因此边上的流量矩阵中,$K$和$L$对应的行一定互为相反数。

网格节点处的函数值通过相邻的网格中心处函数值线性插值得到。
\begin{align*}
u(A) = \sum_{K \in \mathcal{N}(A)} w_{A,K} u(K)
\end{align*}
此时就可以把边界上的流量用单元中心表示出来。对于边$AB$上的流量,需要用到与$A$和$B$相邻的所有单元。

考虑流量表达式中和$u(K),u(L)$有关的项,计算边上的流量时,相当于把边流量矩阵加到总体矩阵中去
\begin{align*}
\frac{|e|}{2}
\left(
\begin{matrix}
\alpha_{K,A} + \alpha_{K,B} & \alpha_{L,A} + \alpha_{L,B} \\
-\alpha_{K,A} - \alpha_{K,B} & -\alpha_{L,A} - \alpha_{L,B}
\end{matrix}
\right)
\end{align*}
它在总体矩阵中对应第$K,L$行和第$K,L$列。

考虑流量表达式中和$u(A)$有关的项。如果节点$A$在区域内部,假设节点$A$相邻的单元编号为$K_1, K_2, \cdots, K_n$。边流量矩阵对应
\begin{align*}
-\frac{|e|}{2} (\alpha_{K,A} + \alpha_{L,A})
\left(
\begin{matrix}
w_{A,K_1} & w_{A,K_2} & \cdots & w_{A,K_n} \\
-w_{A,K_1} & -w_{A,K_2} & \cdots & -w_{A,K_n}
\end{matrix}
\right)
\end{align*}
它在总体矩阵中对应第$K,L$行和第$K_1, K_2, \cdots, K_n$列。

流量表达式中和$u(B)$有关的项和它的计算方法是相同的。

如果节点$A$在区域边界,它直接通过方程的边界条件得到,就是在右端项中$K,L$对应的位置加上
\begin{align*}
\frac{|e|}{2} (\alpha_{K,A} + \alpha_{L,A})
\left(
\begin{matrix}
u(A) \\
-u(A)
\end{matrix}
\right)
\end{align*}

下面考虑区域边界上的边。边$e = AB$是一条位于区域边界的边,它的法向为$n_e$,从单元$K$指向区域外。

如果边上是Dirichelt边界条件,此时边上的流量就是
\begin{align*}
F_{e} = |e| \; (\alpha_{K,A} \, (u(K) - u(A)) + \alpha_{K,B} \, (u(K) - u(B)))
\end{align*}
由于$u(A),u(B)$可以用边界条件直接得到,这相当于直接在总体矩阵的第$K$行第$K$列加上
\begin{align*}
|e| \; (\alpha_{K,A} + \alpha_{K,B})
\end{align*}
然后在第$K$的单元对应的右端项处加上
\begin{align*}
|e| \; (\alpha_{K,A} \, u(A) + \alpha_{K,B} \,  u(B))
\end{align*}

如果边上是Neumann边界条件,此时边上的流量直接就是
\begin{align*}
F_{e} = \int_{e} g(x) \ dx \approx |e| \; g(e)
\end{align*}
用边的长度乘以$g(x)$在边中点上的取值得到。直接在$K$对应的右端项里减去它就可以。

程序里只实现了Dirichlet边界条件。

\section*{五点格式}

五点格式中,我们首先假定单元节点上的函数值已知。我们同样要遍历每条边来装配总体的系数矩阵。

边$e = AB$是一条位于内部的边,它的法向为$n_e$,从单元$K$指向单元$L$。

单元$K$上计算的流量为
\begin{align*}
F_{K, e} = |e| \; (\alpha_1 \, (u(K) - u(P_1)) + \alpha_2 \, (u(K) - u(P_2)))
\end{align*}
其中$P_1, P_2$在$K$的顶点中选择,使联合法向量可以表示成这两个向量的凸组合。$\alpha_1, \alpha_2 > 0$是组合系数。
\begin{align*}
\kappa(K)^T n_{e} = \alpha_1 \, \overrightarrow{K P_1} + \alpha_2 \, \overrightarrow{K P_2}
\end{align*}

同理,在相邻的另一个单元上,我们也可以得到另一侧的流量$F_{L, e}$,对应凸组合的两个点是$P_3, P_4$。注意这里$n_{e}$是从$K$指向$L$的,因此要加负号。$\alpha_3, \alpha_4 > 0$是组合系数。
\begin{align*}
- \kappa(L)^T n_{e} = \alpha_3 \, \overrightarrow{L P_3} + \alpha_4 \, \overrightarrow{L P_4}
\end{align*}

我们可以把两个流加权平均起来,得到最终的流量
\begin{align*}
\mathcal{F}_{e} = & \mu_K \, F_{K, e} - \mu_L \, F_{L, e} 
\end{align*}
其中$\mu_K, \mu_L$是权重。表达式为
\begin{align*}
\mu_K = \frac{t_L}{t_K + t_L}, \quad \mu_L = \frac{t_K}{t_K + t_L}, \quad (t_K + t_L \neq 0). \qquad \mu_K = \mu_L = \frac12, \quad (t_K + t_L = 0).
\end{align*}
其中
\begin{align*}
t_K = \alpha_1 \, u(P_1) + \alpha_2 \, u(P_2), \quad t_L = \alpha_3 \, u(P_3) + \alpha_4 \, u(P_4)
\end{align*}
最终流量为
\begin{align*}
\mathcal{F}_{e} = |e| \left(\mu_K \, (\alpha_1 + \alpha_2) \, u(K) + \mu_L \, (\alpha_3 + \alpha_4) \, u(L) \right)
\end{align*}

最终流量里关于节点函数值的项已经被消掉了,只有和$u(K), u(L)$有关的项。因此在计算边上的流量时,只要把这个边流量矩阵加到总体矩阵中去
\begin{align*}
|e|
\left(
\begin{matrix}
\mu_K \, (\alpha_1 + \alpha_2) & -\mu_L \, (\alpha_3 + \alpha_4) \\
-\mu_K \, (\alpha_1 + \alpha_2) & \mu_L \, (\alpha_3 + \alpha_4) \\
\end{matrix}
\right)
\end{align*}
它在总体矩阵中对应第$K,L$行和第$K,L$列。


边界条件的处理类似。如果边上是Dirichelt边界条件,此时边上的流量就是
\begin{align*}
F_{K, e} = |e| \; (\alpha_1 \, (u(K) - u(P_1)) + \alpha_2 \, (u(K) - u(P_2)))
\end{align*}
由于$u(A),u(B)$可以用边界条件直接得到,这相当于直接在总体矩阵的第$K$行第$K$列加上
\begin{align*}
|e| \; (\alpha_1 + \alpha_2)
\end{align*}
然后在第$K$的单元对应的右端项处加上
\begin{align*}
|e| \; (\alpha_1 \, u(P_1) + \alpha_2 \,  u(P_2))
\end{align*}

对于精确解为正的问题,变量$\alpha_1,\alpha_2,\alpha_3,\alpha_4,\mu_{K},\mu_{L},t_{K},t_{L}$都应该大于$0$,在程序里要仔细检查。

\section*{插值方法}

假设节点$A$相邻的单元编号为$K_1, K_2, \cdots, K_n$,这里通过$K_1, K_2, \cdots, K_n$的单元中心值线性组合近似$A$点处的函数值。这里计算的都是线性插值,我们只需要算出插值的权重$w$就可以了。

平均插值就是权重全部为$1/n$。逆距离插值是把$K_1, K_2, \cdots, K_n$到$A$点距离的倒数归一化之后做为权重。这两种方法都是一阶的,但是保正。

二阶插值的权重是下面这个优化问题的解,我们一方面想让插值和逆距离插值$w_0$接近,另一方面插值要满足二阶条件。
\begin{align*}
\min & \; \|w - w_0\|^2 \\
s.t. & \; A w = b \quad w \geq 0
\end{align*}
其中
\begin{align*}
A = \left(
\begin{matrix}
1 & 1 & \cdots & 1 \\
\frac{\partial u}{\partial x}^{(K_1)} (x_1 - x_A) & \frac{\partial u}{\partial x}^{(K_2)} (x_2 - x_A) & \cdots & \frac{\partial u}{\partial x}^{(K_n)} (x_n - x_A) \\
\frac{\partial u}{\partial y}^{(K_1)} (y_1 - y_A) & \frac{\partial u}{\partial y}^{(K_2)} (y_2 - y_A) & \cdots & \frac{\partial u}{\partial y}^{(K_n)} (y_n - y_A) \\
\end{matrix}
\right)
\quad
b = \left(
\begin{matrix}
1 \\
0 \\
0 \\
\end{matrix}
\right)
\end{align*}

优化问题的解为
\begin{align*}
\lambda = (A \, A^T)^{-1} (A \, w_0 - b) \qquad w = w_0 - A^T \lambda
\end{align*}

扩散系数在整个区域上可能出现间断,这会导致解的梯度不连续,也就是$\nabla u^{(K_i)}$不一定相等。下面给出求$\nabla u^{(K_i)}$的公式。

由于我们事先知道间断线的位置,我们可以在生成网格的时候把网格边放在间断线处。可以假设$\nabla u^{(K_i)}$在单元内部是连续的。设$n_i$是$K_i$和$K_{i+1}$单元之间对应边的法向量,$t_i$是对应的方向向量。

相邻两个单元之间的流连续条件为
\begin{align*}
n_i^T \, (\kappa^{(K_i)} \, \nabla u^{(K_i)}) = n_i^T \, (\kappa^{(K_{i+1})} \, \nabla u^{(K_{i+1})})
\end{align*}
由于解是连续的,解的梯度应该在切向相同
\begin{align*}
t_i^T \, \nabla u^{(K_i)} = t_i^T \, \nabla u^{(K_{i+1})}
\end{align*}

梯度要满足的方程为
\begin{align*}
\left(
\begin{matrix}
n_1^T \kappa^{(K_1)} & -n_1^T \kappa^{(K_2)} &  &  &  \\
t_1^T & -t_1^T &  &  &  \\
 & n_2^T \kappa^{(K_2)} & -n_2^T \kappa^{(K_3)} &  &  \\
 & t_2^T & -t_2^T &  &  \\
 &  & \ddots & \ddots & \ddots \\
-n_n^T \kappa^{(K_1)} &  &  &  & n_n^T \kappa^{(K_n)} \\
-t_n^T &  &  &  & t_n^T
\end{matrix}
\right)
\left(
\begin{matrix}
\nabla u^{(K_1)} \\
\nabla u^{(K_2)} \\
\vdots \\
\nabla u^{(K_n)} \\
\end{matrix}
\right)
= 0
\end{align*}
这里面有$2n$个方程,$2n$个未知数,而且是齐次方程,应该只有零解。但是零解肯定不是我们想要的,这里我们找另一种形式的最小二乘解。
$$ \min_{\|x\| = 1} \|A x\| $$
从这个角度考虑,得到的梯度应该是系数矩阵最小奇异值对应的右奇异向量。

这样就可以算出梯度,进而得到二阶插值。

二阶插值不一定保正。我们的策略是,先用二阶插值试一试,如果得到的结果是保正的,就留下,如果权重中存在小于0的值,就改成逆距离插值。

\end{document}